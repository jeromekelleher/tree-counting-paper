\documentclass{article}

\usepackage{amsmath, amsfonts, amsthm}
\usepackage{algorithm, algpseudocode}

\newcommand{\tskit}{{\texttt{tskit}}}
\newcommand{\halfopen}[2]{[#1, #2)}
\newcommand{\wlw}{\wedge{} \ldots{} \wedge{}}
\newcommand{\nleaves}[1]{|#1|}
\newcommand{\shape}[1]{shape(#1)}
\newcommand{\labels}[1]{labels(#1)}
\newcommand{\smallgap}{{\hspace{0.5em}}}
\newcommand{\multichoose}[2]{\left(\!\!{#1 \choose{} #2}\!\!\right)}

\newtheorem{definition}{Definition}
\newtheorem{remark}{Remark}

\begin{document}

\title{Counting Topologies}
\author{Daniel Goldstein}
\maketitle

\begin{abstract}
    This document outlines the steps taken to efficiently and exactly
    count phylogenetic tree topologies represented in tree sequences.
    The first step is to formulate a reversible hash for leaf-labelled
    trees to efficiently count tree topologies.
    We can obtain this hash using a combinatorial approach, using
    the ranks of trees in the enumeration of leaf-labelled trees.
    The second step uses these tree ranks to count
    topologies inside a large gene tree with a dynamic programming
    approach. This approach works remarkably well for a single tree,
    but proves most effective in its use as part of an incremental
    algorithm on tree sequences, allowing us to fully compute the
    species tree distributions for large numbers of trees in record time.
\end{abstract}

\section{Ranking/Unranking Topologies}
In order to make counting easy we need to define some sort of hash that
exactly describes the space of topologies we are looking to count.

Even though we're mostly interested in bifurcating trees, \tskit{}
represents general trees with any number of children at a node, so we
will too. This is a superset of binary trees so we can still capture
all the same information.

So for some definitions.

\begin{definition}
    Let $S_n$ be the sequence of all unlabelled trees of $n$ leaves.
\end{definition}

\begin{definition}[Shape Rank]
    Let the shape rank of a tree $T$ with $n$ leaves be the index of the
    unlabelled $T$ in $S_n$.
\end{definition}

\begin{definition}
    Let $L_{s;n}$ be the sequence of all labellings of the tree of $n$ leaves
    and shape rank $s$.
\end{definition}

\begin{definition}[Label Rank]
    Let the label rank of a tree $T$ with $n$ leaves and shape rank $s$
    be the index of $T$ in $L_{s;n}$.
\end{definition}

From these definitions we can define the set of all leaf-labelled
series-reduced trees of $n$ leaves to be

\[
    \mathbb{T}_n = \{ T_{s;l} | s \in \halfopen{0}{|S_n|},
                                l \in \halfopen{0}{|L_{s;n}|} \}
\]

where $T_{s;l}$ is the tree with shape rank $s$ and label rank $l$.

\subsection{Enumerating All Tree Topologies}
We must now show how we generate $S_n$ and $L_{s;n}$. To define an enumeration
of trees we will define a canonical orientation, and then a partial ordering.

\subsubsection{Canonical Orientation}
We say that $T = T_0 \wlw{} T_m$ is canonically oriented iff
\[
    \nleaves{T_0} \leq \cdots \leq \nleaves{T_m}
\]
and
\[
    \shape{T_i} = \shape{T_j} \implies \min{\labels{T}} < \min{\labels{T'}}
\]
for all $i,j \in [0, m]$ where $i < j$. Note that the first condition on the ordering
requires the integer partition of $n$ that generates $T$ be an ascending composition.

\subsubsection{Ordering}
Let $T, T' \in S_n$, where $T = T_0 \wlw{} T_m$ and $T' = T'_0 \wlw{} T'_{m'}$.
We define
\[
    T < T' \iff \exists k \in [0, \min(m, m')]
    \smallgap \big{|} \smallgap
    \nleaves{T_k} < \nleaves{T'_k}
    \smallgap \text{and} \smallgap
    \nleaves{T_i} = \nleaves{T'_i} \quad \forall i < k.
\]
% This isn't exactly right because all subtrees could have the same
% number of leaves

to show that the ordering of trees mirrors the ordering of partitions
in the enumeration of ascending compositions. For trees whose subtrees
all have the same number of leaves, the $T < T'$ if there exists
subtrees $T_i < T'_i$.

% Need to extend this ordering to leaf-labelled trees.

\subsubsection{Enumerating}
Discuss here how the canonical orientation maps directly to ascending compositions
+ lexicographic choosing of labels. It also might be nice to note how we can reduce
the tree space to solely binary trees by filtering the ascending compositions to
only size-2 partitions.

\subsection{Ranking}
We define an injective ranking function $R : T \rightarrow \mathbb{N} \times
\mathbb{N}$, such that
\[
    R(T) = (R_s(T), R_l(T)) = (s, l) \implies
    s \in \halfopen{0}{|S_n|}, l \in \halfopen{0}{|L_{s;n}|}
\]

where $s$ and $l$ represent the shape and label ranks of $T$, respectively.
We can easily construct a ranking function to the natural numbers
with the following extension
$R' : \mathbb{N} \times \mathbb{N} \rightarrow \mathbb{N}$:
\[
    R'(s, l) = \sum_{i=0}^{s - 1} |L_{i;n}| + l
\]
but as the following sections rely heavily on the knowledge of $s$ and $l$, we will
mostly forego $R'$ and consider the tuple $(s, l)$ as a rank.

To show that $R$ is a proper ranking function we must define it and show that
\[
    T \neq T' \implies R(T) \neq R(T')
\]
or more powerfully we can show that

\[
    T < T' \implies R(T) < R(T')
\]

This can be broken into two cases:
\begin{enumerate}
    \item $\shape{T} \neq \shape{T'}$ we need only show that $R_s(T) < R_s(T')$.
    \item $\shape{T} = \shape{T'}$ we can see that $R_s(T) = R_s(T')$ and we must
        then show that $R_l(T) < R_l(T')$.
\end{enumerate}

\subsubsection{Defining $R_s$}
First for some partition definitions.

\begin{definition}
    A partition $P_n$ is a sequence of pairs $(p_0, c_0), \ldots, (p_k, c_k)$ such
    that
    \[
        \sum_{i=0}^k p_i c_i = n
    \]
    and $p_0 < \cdots < p_k$. This is known as part-count form of an ascending
    composition, since the parts are increasing.
\end{definition}

\begin{definition}
    Let $P(T_0, \ldots, T_k) \rightarrow P_{\sum_{i=0}^k \nleaves{T_i}}$
    be a function that given a sequence of trees returns the integer partition that
    describes the number of leaves in each tree in part-count form.
\end{definition}

Let $T = T_0 \wlw T_m \in S_n$, and let $P(T_0, \ldots, T_m) = P_n^j$ be the $j$th integer
partition of $n$.
We can define $R_s(T)$ as
\[
    R_s(T) = \sum_{i=0}^{j} f(P_n^i) + \sum_{i=0}^m R_s(T_i)f(P(T_{i+1}, \ldots, T_m))
\]

% better symbol for f
where $f$ is the function that produces the number of unique tree shapes possible
given a partition.
\[
    f(P_n)
    = f((p_0, c_0), \ldots, (p_k, c_k))
    = \prod_{i=0}^k \multichoose{|S_{p_i}|}{c_i}
\]

We can now show case \#1,
$T < T' \land \shape{T} \neq \shape{T'} \implies R_s(T) < R_s(T')$.

Let $T, T' \in S_n$, where $T < T'$ and $\shape{T} \neq \shape{T'}$.
Here are faced with two cases.
The first is that $T$ and $T'$ were generated by two different
partitions, $P_n$ and $P_n'$, where $P_n$ comes before $P_n'$
in the enumeration of ascending compositions
(might need to show that separately, but it can be easily derived
from the canonical orientation).
By inspecting $R_s$, we can see that the all trees generated by $P_n$
are counted in the first term. We can more rigorously show that by
showing how

\begin{equation}
    \sum_{i=0}^m R_s(T_i)f(P(T_{i+1}, \ldots, T_m)) < f(P(T_0, \ldots, T_m))
\end{equation}

and expanding out the $f$ calls, and $R_s(T) < |S_n|$ when $T \in S_n$.

The second case is when $T$ and $T'$ were both generated by the same
partition $P_n$. In this case, by the canonical ordering we know that
there must be subtrees $T_j$ and $T'_j$ such that $T_j < T'_j$. By
induction we can safely say that $R_s(T_j) < R_s(T_j')$. Now we can see
that
\begin{align*}
    R_s(T)
    &= \sum_{i=0}^{j} f(P_n^i) + \sum_{i=0}^m R_s(T_i)f(P(T_{i+1}, \ldots, T_m))\\
    &= \sum_{i=0}^{j} f(P_n^i) + \sum_{i=0}^{j-1} R_s(T_i)f(P(T_{i+1}, \ldots, T_m))\\
    &+ R_s(T_j)f(P(T_{j+1}, \ldots, T_m)) + \sum_{i=j+1}^m R_s(T_i)f(P(T_{i+1}, \ldots, T_m))\\
    &< \sum_{i=0}^{j} f(P_n^i) + \sum_{i=0}^{j-1} R_s(T_i)f(P(T_{i+1}, \ldots, T_m))\\
    &+ R_s(T'_j)f(P(T_{j+1}, \ldots, T_m))\\
    &< \sum_{i=0}^{j} f(P_n^i) + \sum_{i=0}^{j-1} R_s(T_i)f(P(T_{i+1}, \ldots, T_m))\\
    &+ R_s(T'_j)f(P(T_{j+1}, \ldots, T_m)) + \sum_{i=j+1}^m R_s(T'_i)f(P(T'_{i+1}, \ldots, T'_m))\\
    &= R_s(T')
\end{align*}
using Eq 1 to show that
$f(P(T_{j+1}, \ldots, T_m)) > \sum_{i=j+1}^m R_s(T_i)f(P(T_{i+1}, \ldots, T_m))$, and
relying on the fact that $P(T_i, \ldots, T_m) = P(T'_j, \ldots, T'_m)$.

\subsubsection{Defining $R_l$}

\subsection{Unranking}
For this probably best to split it up into something like
$U(s, l, n) = U_l(U_s(s, n), l)$.

\end{document}
